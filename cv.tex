\documentclass{article}

\usepackage[utf8]{inputenc}
\usepackage[margin=1.0cm]{geometry}
\usepackage[english]{babel}
\usepackage{array,xcolor}
\usepackage{hyperref}
\usepackage{multicol}

\definecolor{lightgray}{gray}{0.8}
\newcolumntype{L}{>{\raggedleft}p{0.20\textwidth}}
\newcolumntype{R}{p{0.75\textwidth}}
\newcommand\VRule{\color{lightgray}\vrule width 0.5pt}

\setlength{\parindent}{0pt}
\pagenumbering{gobble}

\newcommand\jobtitle\textit

\title{\vspace{-1.5em}\textbf{Jacob Thomas Errington}}
\author{%
  \texttt{jake@mail.jerrington.me} -- \url{https://jerrington.me} \\
  $+1\;(514)\;503-3100$%
}
\date{}

\begin{document}

\maketitle

\hrule

\subsection*{Work experience}

\begin{tabular}[h]{L!{\VRule}R}
  Aug. 2020 to present
  & \jobtitle{Software developer} @ DRW Trading Group \\
  & Full-stack web development (NodeJS, React, TypeScript). \\
  & High-performance market data processing (C++, libuv, RabbitMQ). \\
  & Participate in design, communicate with stakeholders, produce and review
    code. \\
  & Community: presenting at quarterly lightning talks,
    active in knowledge-sharing in the team. \\
  \\
  Fall 2019
  & \jobtitle{Course instructor} @ McGill University \\
  & Co-teach COMP~302, a second-year course about functional
    programming. ($\sim 150$ students.)\\
  & Give lectures, prepare and grade exams, supervise teaching assistants, hold office
    hours. \\
  \\
  Multiple times
  & \jobtitle{Teaching assistant} @ McGill University \\
  & Fall 2017, Fall 2018: COMP 302 (functional programming) \\
  & Winter 2019: COMP 527 (logic and computation) \\
  & Advise students on homework problems, participate in grading homework
    and exams, assist in preparing homework and in course administration. \\
  \\
  %
  % May 2017 to May 2018
  % & \jobtitle{Technical lead} (part-time)
  %   @ Building21, Montreal; \url{http://building21.ca} \\
  % & Advise the purchase of new equipment, maintain existing PCs, web
  %   development. \\
  % \\
  Jun. 2017 to Aug. 2017
  & \jobtitle{Software developer intern}
    @ 1010data Services Inc., New York City; \url{http://1010data.com} \\
  & Linux systems programming in C, application programming in
    \href{https://en.wikipedia.org/wiki/K_\%28programming_language\%29}{K}.
  \\
  & Improved login time by a factor of 12. \\
  & Developed new infrastructure for debugging K applications. \\
  & \emph{Received post-graduation return offer.} \\
  \\
  Feb. 2016 to Feb. 2017
  & \jobtitle{Software engineer}
    @ (formerly) OOHLALA Mobile; \url{https://www.readyeducation.com} \\
  & Backend web and server-side development; Python, SQL. \\
  & Develop integrations with existing institutional infrastructure. \\
    % & \\
  % Jan. 2015 to Sep. 2015
  %   & \jobtitle{CTO and cofounder}
  %   @ Ahvoda Recruitment Inc., Montreal \\
  %   & Part of the McGill X-1 startup accelerator,
  %     \url{http://mcgillx1accelerator.com} \\
  %   & Full-stack web development, native Android development,
  %     database administration. \\
  %   & \\
  % Jun. 2014 to Jun. 2015
  %   & \jobtitle{Research assistant}
  %   @ McGill University and Genome Quebec Innovation Centre \\
  %   & Scientific Python programming in genetics, big data. \\
  %   & Supervised by
  %     \href{http://simongravel.lab.mcgill.ca/Home.html}{Simon Gravel},
  %     \href{mailto:simon.gravel@mcgill.ca}{simon.gravel@mcgill.ca}
\end{tabular}

\subsection*{Education}

\begin{tabular}[h]{L!{\VRule}R}
  May 2018 to Sep. 2020
  & \jobtitle{M.Sc. Computer Science (Thesis)} @ McGill University \\
  & Part of the \href{http://cs.mcgill.ca/~complogic}{Computation and Logic
    Group} led by \href{http://www.cs.mcgill.ca/~bpientka/}{Brigitte Pientka}. \\
  & Research on interactive theorem proving, proof automation. \\
  & Funding by FRQNT B1X scholarship (\$35,000 value) \\
  & CGPA $3.91$ \\
  \\
  Sep. 2014 to Dec. 2017
  & \jobtitle{B.Sc. Mathematics \& Computer Science} @ McGill University \\
  & Relevant courses: compiler design, computability theory, logic,
    set theory. \\
  & CGPA $3.74$ \\
  % 2012 to 2014
  %   & \jobtitle{D\'EC Pure and Applied Science} (Honours),
  %     Dawson College, Westmount. \\
  %   & Admitted to the Honour Roll every semester. Final average above 90\%.
\end{tabular}

% \subsection*{Awards}
%
% \begin{tabular}[h]{L!{\VRule}R}
%   May 2018 to May 2020
%   & \jobtitle{FRQNT B1X Scholarship} (\$35,000 value) \\
%   & Awarded to M.Sc. students for academic excellence. \\
%   \\
%   Jan. 2018 to Apr. 2018
%   & \jobtitle{NSERC USRA} (\$5,625 value) \\
%   & Awarded for undergraduate research project with Prof.~B.~Pientka. \\
% \end{tabular}

% \subsection*{Volunteer \& Outreach}
%
% \begin{tabular}[h]{L!{\VRule}R}
%   Sep. 2015 to May 2016
%     & \jobtitle{Outreach director} @ HackMcGill, McGill University's hacker club. \\
%     & Organized community events and tutorials,
%       participated in organizing \href{http://mchacks.io/}{McHacks}.
% \end{tabular}

\subsection*{Projects}

\begin{description}
  \item[\href{https://computing-workshop.com/}{Computing Workshop.}]
    My partner and I create and run small workshops on computer science.
    We have prepared units on fundamentals of circuits, functional
    programming, machine learning, and more. Each unit is six, two-hour
    sessions in duration.

  \item[Haskell libraries.]
    I maintain a set of
    \href{https://github.com/tsani/pushbullet-hs}{Haskell libraries} for
    interacting with the \href{https://pushbullet.com}{PushBullet} API.
    I also maintain
    \href{https://github.com/tsani/servant-github-webhook}{servant-github-webhook},
    a library to securely integrate GitHub webhooks with the Servant REST
    framework.

  % \item[Fast accumulator startup.]
  %   In a 10-week internship at 1010data in New York City, I worked to
  %   accelerate the core platform login.
  %   My work achieves this by introducing a caching mechanism:
  %   new user session processes are now started via forking.
  %   This mechanism is built as a C extension to the K interpreter, and includes
  %   a framework for reparenting terminal applications to facilitate debugging.
\end{description}

\end{document}
