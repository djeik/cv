\documentclass{article}

\usepackage[utf8]{inputenc}
\usepackage[margin=1.0cm]{geometry}
\usepackage[english]{babel}
\usepackage{array,xcolor}
\usepackage{hyperref}
\usepackage{multicol}

\definecolor{lightgray}{gray}{0.8}
\newcolumntype{L}{>{\raggedleft}p{0.20\textwidth}}
\newcolumntype{R}{p{0.80\textwidth}}
\newcommand\VRule{\color{lightgray}\vrule width 0.5pt}

\setlength{\parindent}{0pt}
\pagenumbering{gobble}

\newcommand\jobtitle\textit

\title{\vspace{-1.5em}\textbf{Jacob Thomas Errington}}
\author{\texttt{jake@mail.jerrington.me} -- \url{https://jerrington.me} \\ $+1\;(514)\;503-3100$}
\date{}

\begin{document}

\maketitle

\hrule

\subsection*{Education}

\begin{tabular}[h]{L!{\VRule}R}
  2018 to present
    & \jobtitle{M.Sc. Computer Science}, McGill University, Montreal, Quebec \\
    & Part of the \href{http://cs.mcgill.ca/~complogic}{Computation and Logic
    Group} led by \href{http://www.cs.mcgill.ca/~bpientka/}{Brigitte Pientka}. \\
    & Research on proof search algorithms and interactive program development. \\
  2014 to 2017
    & \jobtitle{B.Sc. Mathematics \& Computer Science}, McGill University,
      Montreal, Quebec \\
    & CGPA $3.74$ \\
    & Relevant courses: compiler design, computability theory, logic,
      set theory. \\
  2012 to 2014
    & \jobtitle{D\'EC Pure and Applied Science} (Honours),
      Dawson College, Westmount. \\
    & Admitted to the Honour Roll every semester. Final average above 90\%.
\end{tabular}

\subsection*{Work experience}

\begin{tabular}[h]{L!{\VRule}R}
  May 2017 to May 2018
    & \jobtitle{Technical lead} (part-time)
      @ Building21, Montreal; \url{http://building21.ca} \\
    & Advising the purchase of new equipment, maintaining existing PCs, web
      development. \\
    & \\
  Jun. 2017 to Aug. 2017
    & \jobtitle{Software developer intern}
      @ 1010data Services Inc., New York City; \url{http://1010data.com} \\
    & Linux systems programming in C, application programming in
      \href{https://en.wikipedia.org/wiki/K_\%28programming_language\%29}{K}.
      \\
    & Improved login time by a factor of 12. \\
    & Developed new infrastructure for debugging K applications. \\
    & \emph{Received post-graduation return offer.} \\
    & \\
  Feb. 2016 to Feb. 2017
    & \jobtitle{Software engineer}
    @ OOHLALA Mobile Inc., Montreal; \url{http://oohlalamobile.com} \\
    & Backend web and server-side development; Python, SQL. \\
    & Developed integrations with existing institutional infrastructure. \\
    & \\
  Jan. 2015 to Sep. 2015
    & \jobtitle{CTO and cofounder}
    @ Ahvoda Recruitment Inc., Montreal \\
    & Part of the McGill X-1 startup accelerator,
      \url{http://mcgillx1accelerator.com} \\
    & Full-stack web development, native Android development,
      database administration. \\
    & \\
  Jun. 2014 to Jun. 2015
    & \jobtitle{Research assistant}
    @ McGill University and Genome Quebec Innovation Centre \\
    & Scientific Python programming in genetics, big data. \\
    & Supervised by
      \href{http://simongravel.lab.mcgill.ca/Home.html}{Simon Gravel},
      \href{mailto:simon.gravel@mcgill.ca}{simon.gravel@mcgill.ca}
\end{tabular}

\subsection*{Volunteer \& Outreach}

\begin{tabular}[h]{L!{\VRule}R}
  Sep. 2015 to May 2016
    & Outreach director at HackMcGill, McGill University's hacker club. \\
    & Organized community events and tutorials,
      participated in organizing \href{http://mchacks.io/}{McHacks}.
\end{tabular}

\subsection*{Projects}

\begin{description}
  \item[\href{https://computing-workshop.com/}{Computing Workshop.}]
    My partner and I create and run small workshops on computer science.
    We have prepared units on fundamentals of circuits, functional
    programming, and machine learning and more. Each unit is six, two-hour
    sessions in duration.
    
  \item[Fast accumulator startup.]
    In a 10-week internship at 1010data in New York City, I worked to
    accelerate the core platform login.
    My work achieves this by introducing a caching mechanism:
    new user session processes are now started via forking.
    This mechanism is built as a C extension to the K interpreter, and includes
    a framework for reparenting terminal applications to facilitate debugging.

  % \item[Goto.]
  %    For a course on compiler design, my teammate and I implemented a compiler
  %    for a subset of Go called GoLite in Haskell. Our compiler featured a
  %    partial x86\_64 code generator, as well as a complete C++ code generator.
  %    Our C++ generator emitted faster code than the reference compiler.

  %    As McGill does not have an A+ grade, Professor Laurie Hendren contacted me
  %    to award me with an honorary A+ for outstanding achievement in the course.
\end{description}

See \url{https://jerrington.me/projects.html} for more.

\end{document}
